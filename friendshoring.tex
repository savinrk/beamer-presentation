\documentclass[10pt,xcolor={dvipsnames}]{beamer}
\usepackage{pgfpages,epigraph, enumitem, appendixnumberbeamer, fontspec, pifont, adjustbox, amsfonts, amsmath, amssymb, svg, graphicx, csquotes, caption, subcaption, booktabs, tabularx, transparent}

% \usefontstheme{professionalfonts}
% 

\usefonttheme{professionalfonts}

% \renewcommand\mathfamilydefault{\rmdefault}
\usepackage[english]{babel}
\usepackage[utf8x]{inputenc}

\usetheme{metropolis}
\usecolortheme{beaver} 

\setbeamertemplate{footline}[text line]{%
  \parbox{\linewidth}{\vspace*{-10pt} Khadka, Gopinath, and Batarseh (2023) - \textit{Friendshoring in Agricultural Trade} \hfill 2023 IATRC Annual Meeting \hspace{0.3cm}\insertframenumber \hspace{0.1em} /  \inserttotalframenumber}}
\setbeamertemplate{author}{\centering \textsc{\insertauthor} \par}
\setbeamertemplate{date}{\scriptsize \vspace{1cm} \centering \insertdate \par}
\setbeamertemplate{institute}{ \raggedleft \vspace{1cm}  \textsf{\scriptsize \insertinstitute} \par}
\setlist[itemize]{label={\scalebox{0.8}{$\diamond$}}, itemsep = 10pt}
\setsansfont[BoldFont={Fira Sans SemiBold}]{Fira Sans Book}

\title{Shifting Alliances - Friendshoring in Agricultural Trade}
\author[Khadka, Gopinath, and Batarseh (2023)]{\scriptsize Savin Khadka \inst{1} \and Munisamy Gopinath, PhD \inst{1} \and  Feras A. Batarseh, PhD \inst{2}}
\institute{\scriptsize \inst{1} Dept. of Ag. and Applied Economics \\ University of Georgia \and  \inst{2} Dept. of Biological Systems Engineering \\ Virginia Tech. }
\scriptsize \date{ 2023 IATRC Annual Meeting \\ 
Dec 10-12, Clearwater Beach FL}



\begin{document}

\begin{frame}[plain]
  \titlepage
\end{frame}

% Uncomment these lines for an automatically generated outline
%
\begin{frame}[plain]{Outline}
  \tableofcontents
\end{frame}
\section{Background}

\begin{frame}[plain]{Uncertainties}
  \begin{columns}[T] % align columns
  \scriptsize
    \begin{column}{.48\textwidth}
               \begin{itemize}\itemsep20pt
        \item[\ding{213}] Uncertainty about the future has significant impact on the economy - investment, output, trade ... 
        \item[\ding{213}] Uncertainty can arise from various sources -  conflicts (political/armed/trade), economic, ...  \pause
        \item[\ding{213}] Previous work has shown that policy uncertainty in particular has large effects on international trade. 
        \item[\ding{213}] Machine Learning vs. traditional methods ...
    \end{itemize}
    \end{column}% 
    \pause
    \hfill
    \begin{column}{.7\textwidth}
        \includegraphics[width=\textwidth]{figs/presentationFigs/jaaea.png}
    \end{column}%
\end{columns}
\end{frame}


\begin{frame}{Friendshoring}
    \begin{itemize}\itemsep3.4ex
        \item[\ding{213}] \textbf{Three Cs: consequences and considerations}
            \begin{itemize}\itemsep1ex
                    \item \textcolor{NavyBlue}{COVID} - Severe disruptions to the supply/demand sides 
                    \item \textcolor{red}{Conflict} - Escalating political and armed disputes  
                    \item \textcolor{ForestGreen}{Climate Change} - Ongoing discussions about the future of production and distribution
                   
            \end{itemize} \pause
        \item[\ding{213}] \textbf{Additional uncertainties} $\implies$ Restructuring of supply chains: rivals $\rightarrow$ allies \pause
        \item[\ding{213}] \textbf{Resiliency vs. Efficiency/Costs}
            \begin{itemize}\itemsep1ex
                \item $\approx$ 5\% drop in global production (WTO)
                \item $\approx$ 2\% drop in global economic output (IMF)
            \end{itemize}
    \end{itemize}
\end{frame}
\begin{frame}[plain]{Background}
\begin{columns}[T] % align columns
    \begin{column}{.48\textwidth}
                \vspace{-0.cm}
                \includegraphics[scale=0.1]{figs/presentationFigs/test.pdf} \par
                \vspace{10pt}
                \includegraphics[scale=0.1]{figs/presentationFigs/wsj-ChinaHabit.png}\par
                \vspace{10pt}
                \includegraphics[scale=0.2]{figs/presentationFigs/wsj-chinaTies.png}
    \end{column}% 
    \pause
    \hfill
    \begin{column}{.48\textwidth}
        \vspace{0.25in}
        \footnotesize ``Rather than being highly reliant on countries
                        where we have geopolitical tensions and can’t count on ongoing, reliable supplies, we need to really diversify our group of suppliers. '' \par
        \small \textit{  Janet Allen, US Treasury Secretary (2022)} \par 
        \vspace{0.25in}
        \footnotesize ``We have no eternal allies, and we have no                         perpetual enemies. Our interests are eternal and                  perpetual, and those interests it is our duty to follow. '' \par
        \small \textit{Lord Palmerston, UK Prime Minister  (1848)}
    \end{column}%
\end{columns}
\end{frame}

\section{Data and Methods}
\begin{frame}{Data}

\begin{table}[H]
    \centering
    \caption*{\small Data Summary}
    \vspace{-0.2in}
    \scriptsize
   \resizebox{\textwidth}{!}{
   \begin{tabular}{cccc}
        \toprule
         \textbf{Product}&  \textbf{HS4-Code} & \textbf{2021 Trade Value (billions of USD)}  & \textbf{Source}\\
        \midrule
        \textbf{\textit{Meats}}\\
         Fresh/chilled Beef & HS-0201 & 28.8 & IHS Markit \\
         Frozen Beef & HS-0202 & 32.8 & IHS Markit \\
         Pork & HS-0203 & 36.9 & IHS Markit \\
         Sheep/Goat & HS-0204 & 10.5 & IHS Markit\\
         Chicken & HS-0207 & 30.6 & IHS Markit \\
         \\
         \textbf{\textit{Grains and Legumes}}\\
         Wheat & HS-1001 & 61.8 & UN-Comtrade \\
         Corn & HS-1005 & 52.3 & IHS Markit \\
         Rice & HS-1006 & 28.4 & UN-comtrade \\
         Soybean & HS-1201 & 78.5 & IHS Markit \\
         \\
         \textbf{\textit{Edible Oils}}\\
         Soybean Oil & HS-1507 & 17.1 & IHS Markit \\
         Peanut Oil& HS-1508 & 0.713 & IHS Markit \\
         Palm Oil& HS-1511 & 51.1 & IHS Markit \\
         Sunflower Oil& HS-1512 & 16.7 & IHS Markit \\
         Rapeseed Oil & HS-1514 & 12.3 & IHS Markit \\
        \bottomrule 
    \end{tabular}}
    
\end{table}

\end{frame}


\begin{frame}{Network Approach}
    \begin{figure}
        \centering
         \includegraphics[width=0.8\textwidth]{figs/presentationFigs/corn2022.png}
         \caption*{\tiny  Corn Trade Network for 2022}
    \end{figure}
% \hfill
\vspace{-20pt}
\footnotesize
%     \begin{column}{.3\textwidth}
         \begin{itemize}\itemsep1pt
             \item[\ding{213}] Nodes (countries) and Edges (trade) 
             \item[\ding{213}] Networks dynamics for analyses and prediction 
             \begin{itemize}
             \scriptsize
                 \item \textit{Centrality Measures, Community Detection, Global Clustering}
             \end{itemize}
         \end{itemize}
%     \end{column}
% \end{columns}
\end{frame}
\begin{frame}{Centrality Measures}
    \begin{itemize}\itemsep15pt
        \small
        \item[\ding{213}] \textbf{Degree Centrality}
            \scriptsize
            \begin{itemize}\itemsep0.4pt
                \item Measures connections to other nodes 
                \item Higher degree centrality $\implies$ trades with \textcolor{blue}{more partners}
                \item eg: more friends on Facebook 
                \item In directed networks, in-degree = imports partners \& out-degree = exports partners
                \item $D_i = \sum_j e(i,j)$, where $e(i,j) = 1$ if link present, $0$ otherwise
            \end{itemize}
            
        \small
        \item[\ding{213}] \textbf{Betweenness Centrality}
            \scriptsize
            \begin{itemize}\itemsep0.4pt
                \item Measures connections facilitated between other nodes
                \item Higher betweenness centrality $\implies$ acts like a \textcolor{blue}{bridge}
                \item eg: mutual friends with lots of people
                \item $B_i = \sum_{a,b} \frac{n_{aib}}{n_{ab}} = $  fraction of optimal paths between $a$ and $b$ passing through $i$
            \end{itemize}
        
        \small
        \item[\ding{213}] \textbf{Laplacian Centrality}
            \scriptsize
            \begin{itemize}\itemsep0.4pt
                \item Measures \textcolor{blue}{global influence} on the network
                \item Higher Laplacian centrality $\implies$ large change in the network if removed
                \item eg: friends with \textit{popular} people
                \item $L_i = \frac{E_L(G) - E_L(G_i)}{E_L(G)}$, $E_L(G) = $ Laplacian Energy \& $E_L(G_i)$ Laplacian Energy with $i$ removed
            \end{itemize}
    \end{itemize}
\end{frame}

\begin{frame}{Community Detection}
    \begin{itemize}\itemsep20pt
        \small
        \item[\ding{213}] Community = groups of nodes that are densely interconnected.
        \item[\ding{213}] More connections within communities and few between communities.
        \item[\ding{213}] Identified by maximizing \textit{\textbf{modularity}}: \scriptsize (Zhu, Chen, and Zeng, 2020)\\
         \small \centering $ Q =  \sum_{c=1}^{N} \left [ \frac{\textcolor{ForestGreen}{L_c}}{\textcolor{blue}{m}} - \textcolor{Sepia}{\lambda} \left( \frac{\textcolor{orange}{k_c^{in} k_c^{out}}}{2\textcolor{blue}{m}}\right) \right]$,\\
        \scriptsize 
        \raggedright
            $\textcolor{ForestGreen}{L_c} = $ number of intra-community links in community $c$,\\
            $\textcolor{blue}{m} = $ number of edges in community, \\
            $\textcolor{Sepia}{\lambda}= $ resolution limit, \\
            $\textcolor{orange}{k_c^{in}, k_c^{out}} = $ sums of in- and out-degrees of nodes in community $c$, \\
        
        \small
        \item[\ding{213}] To detect optimal communities for each commodity for each year:    
        
        \begin{itemize}\itemsep2pt
            \scriptsize
            \item assign each node to its own community
            \item join pairs of communities such that modularity is maximized
            \item conclude when no further modularity increase is possible
        \end{itemize}
    \end{itemize}  
\end{frame}



\begin{frame}{Clustering}
    \begin{itemize}\itemsep20pt
        \small
        \item[\ding{213}] Measures tendency of connectivity within a network.
        \item[\ding{213}] Function of \textit{triangles} observed in a network vs. total \textit{triangles} possible.\\
        \item[\ding{213}] For a weighted directed network, clustering coefficient of each node $i$: \\
          \phantom{sakjflsajfdlksjf} $i_c = \frac{N_i}{2(deg^{tot}(i)(deg^{tot}(i)-1) - 2deg^{false}(i))} \hfill \frac{\textcolor{blue}{ \Delta \text{s observed}}}{\textcolor{blue}{\text{total } \Delta \text{s possible}}}$  \par
        \tiny
        $N_i = $ number of directed triangles through node $i$,\\
        $deg^{tot}_i = $ sum of in- and out-degrees of node $i$, \\
        $deg^{false}_i = $ number of ``false'' triangles through node $i$

        \small
        \item[\ding{213}] The global clustering coefficient of the whole network is the average of clustering coefficient for all nodes: \\
        \small \centering $ \text{\footnotesize Average Clustering Coefficient} (G_c) = \frac{1}{I}\sum_i^I t_i, $ \\
        \raggedright 
        \tiny
        $I = $ number of nodes in the network, \\
        $t_i = $ clustering coefficient for each node $i$\\
        
    \end{itemize}
\end{frame}

\section{Results}
\begin{frame}{Centrality Changes - Pork}
    \begin{figure}
        \centering
        \caption*{\footnotesize Centrality Changes for Pork}
        \includegraphics[width=\textwidth]{figs/presentationFigs/porkCentralities.pdf}
    \end{figure}
    \begin{itemize}
        \small
        \item[\ding{213}] China now imports from more partners. 
        \item[\ding{213}] US exporting to fewer partners, Russian imports have collapsed. 
        \item[\ding{213}] Betweenness decreasing for all countries. 
        \item[\ding{213}] China's influence on global network is remarkably higher compared to the US.  
    \end{itemize}
\end{frame}
\begin{frame}[plain]{Community Detection - Pork}
\resizebox{\textwidth}{!}{
        \centering
        \begin{tabular}{c|cccccc}
        \toprule
            & \multicolumn{6}{c}{\textbf{Average Number of Communities }} \\
        \hline
          \textbf{Commodity}   & \textbf{1995-1999} & \textbf{2000-2004} & \textbf{2005-2009} & \textbf{2010-2014}& \textbf{2015-2019} & \textbf{\textcolor{blue}{2020-2022}} \\
          \hline
          Pork & 6.6 & 6.6 & 6.4  & 7.6 & 8.4 & \textcolor{blue}{8.67}\\
        \bottomrule
        \end{tabular}
        \caption{Average Number of Communities}
 }
\begin{itemize}
 \footnotesize
    \item[\ding{213}] Number of communities increasing $\implies$ network more divided
\end{itemize}
\pause

\includegraphics[width=\textwidth]{figs/presentationFigs/pork-comms-table.png}

\end{frame}

\begin{frame}{Communities - Pork (2013)}
    \includegraphics[width=\textwidth]{figs/networks/pork2013network.png}
\end{frame}

\begin{frame}{Communities - Pork (2019)}
    \includegraphics[width=\textwidth]{figs/presentationFigs/pork2019all.png}
\end{frame}


\iffalse
\begin{frame}{Communities - Pork (China)}

\begin{columns}[onlytextwidth]
    \begin{column}{.5\textwidth}
        \begin{minipage}[t][.5\textheight][t]{\textwidth}
            \begin{figure}
                \caption*{\tiny China neighbors (2017)}
                \includegraphics[scale=0.1]{figs/networks/pork2017chinaNeighbors.png}
            \end{figure}
        \end{minipage}
    \end{column}
\hfill
    \begin{column}{.5\textwidth}
        \begin{minipage}[t][.5\textheight][t]{\textwidth}
            \begin{figure}
                \caption*{\tiny China neighbors (2019)}
                \includegraphics[scale=0.1]{figs/networks/pork2019chinaNeighbors.png}\\
            \end{figure}
        \end{minipage}
    \end{column}
\end{columns}
\vspace{-35pt}
\begin{figure}
    \centering
    \caption*{\tiny China neighbors (2022)}    
    \includegraphics[scale=0.15]{figs/networks/pork2022chinaNeighbor.png}    
\end{figure}

\end{frame}
\begin{frame}{Communities - Pork (US)}

\begin{columns}[onlytextwidth]
    \begin{column}{.5\textwidth}
        \begin{minipage}[t][.5\textheight][t]{\textwidth}
            \begin{figure}
                \caption*{\tiny US neighbors (2017)}
                \includegraphics[scale=0.1]{figs/networks/pork2017usneighbors.png}
            \end{figure}
        \end{minipage}
    \end{column}
\hfill
    \begin{column}{.5\textwidth}
        \begin{minipage}[t][.5\textheight][t]{\textwidth}
            \begin{figure}
                \caption*{\tiny US neighbors (2019)}
                \includegraphics[scale=0.1]{figs/networks/pork2019usneighbors.png}\\
            \end{figure}
        \end{minipage}
    \end{column}
\end{columns}
\vspace{-35pt}
\begin{figure}
    \centering
    \caption*{\tiny US neighbors (2022)}
    \includegraphics[scale=0.11]{figs/networks/pork2022usbeinghbors.png}    
\end{figure}

\end{frame}
\begin{frame}{Communities - Pork (Brazil)}
\begin{columns}[onlytextwidth]
    \begin{column}{.5\textwidth}
        \begin{minipage}[t][.5\textheight][t]{\textwidth}
            \begin{figure}
                \caption*{\tiny Brazil neighbors (2017)}
                \includegraphics[scale=0.1]{figs/networks/pork2017brazilneighbors.png}
            \end{figure}
        \end{minipage}
    \end{column}
\hfill
    \begin{column}{.5\textwidth}
        \begin{minipage}[t][.5\textheight][t]{\textwidth}
            \begin{figure}
                \caption*{\tiny Brazil neighbors (2019)}
                \includegraphics[scale=0.1]{figs/networks/pork2019brazilneighbors.png}\\
            \end{figure}
        \end{minipage}
    \end{column}
\end{columns}
\vspace{-35pt}
\begin{figure}
    \centering
    \caption*{\tiny Brazil neighbors (2022)}

    \includegraphics[scale=0.15]{figs/networks/pork2022brazilneighbors.png}    
\end{figure}

\end{frame}
\fi 

\begin{frame}{Clustering - Pork}
\begin{figure}
    \centering
    \includegraphics[scale=0.7]{figs/presentationFigs/pork-clust.pdf}
\end{figure}
\begin{itemize}
    \small
    \item[\ding{213}] Declining clustering coefficient $\implies$ decline in network stability \\
    \hfill \footnotesize (less interconnected than before)\quad \phantom{asdj}
    \small
    \item[\ding{213}] Same pattern is observed in other food commodities.
\end{itemize}
\end{frame}

\begin{frame}{Centrality Changes - Corn}
    \begin{figure}
        \centering
        \caption*{\footnotesize Centrality Changes for Corn}
        \includegraphics[width=\textwidth]{figs/centralities/corn-cent.pdf}
    \end{figure}
    \begin{itemize}
        \small
        \item[\ding{213}] China's now imports from more partners. 
        \item[\ding{213}] US not exports to fewer partners. 
        \item[\ding{213}] Betweenness decreasing for all countries, US trends upwards after Phase One agreement and COVID recovery. 
        \item[\ding{213}] US's influence on global network is declining.  
    \end{itemize}
\end{frame}

\begin{frame}[plain]{Community Detection - Corn}
\resizebox{\textwidth}{!}{
        \centering
        \begin{tabular}{c|cccccc}
        \toprule
            & \multicolumn{6}{c}{\textbf{Average Number of Communities}} \\
        \hline
          \textbf{Commodity}   & \textbf{1995-1999} & \textbf{2000-2004} & \textbf{2005-2009} & \textbf{2010-2014}& \textbf{2015-2019} & \textbf{\textcolor{blue}{2020-2022}} \\
          \hline
          Corn & 7.0 & 7.2 & 6.6 & 7.2 & 6.8 & \textcolor{blue}{7.33}\\
        \bottomrule
        \end{tabular}
        \caption{Average Number of Communities}
 }


\includegraphics[width=\textwidth]{figs/presentationFigs/corn-comm-tables.png}
\end{frame}

\begin{frame}{Communities - Corn (2013) }
\includegraphics[width=\textwidth]{figs/presentationFigs/corn2013all.png}\\
% \scriptsize \phantom{lsakdjflsdjflksa} Corn Communities (2013) \phantom{aksjldflkajdlkfsafdjjdf} Corn Communities (2019) 
% \phantom{sfjlaskjfasjdfksdjlfa}



\end{frame}

\begin{frame}{Communities - Corn (2019)}
    \includegraphics[width=\textwidth]{figs/presentationFigs/corn2019all.png}
\end{frame}

\iffalse
\begin{frame}{Communities - Corn (China)}

\begin{columns}[onlytextwidth]
    \begin{column}{.5\textwidth}
        \begin{minipage}[t][.5\textheight][t]{\textwidth}
            \begin{figure}
                \caption*{\tiny China neighbors (2017)}
                \includegraphics[scale=0.15]{figs/networks/corn2017chinaneighbors.png}
            \end{figure}
        \end{minipage}
    \end{column}
\hfill
    \begin{column}{.5\textwidth}
        \begin{minipage}[t][.5\textheight][t]{\textwidth}
            \begin{figure}
                \caption*{\tiny China neighbors (2019)}
                \includegraphics[scale=0.13]{figs/networks/corn2019chinaneighbors.png}\\
            \end{figure}
        \end{minipage}
    \end{column}
\end{columns}
\vspace{-35pt}
\begin{figure}
    \centering
    \caption*{\tiny China neighbors (2022)}
    \includegraphics[scale=0.15]{figs/networks/corn2022chinaneighbor.png}    
\end{figure}
\end{frame}
\begin{frame}{Communities - Corn (US)}

\begin{columns}[onlytextwidth]
    \begin{column}{.5\textwidth}
        \begin{minipage}[t][.5\textheight][t]{\textwidth}
            \begin{figure}
                \caption*{\tiny US neighbors (2017)}
                \includegraphics[scale=0.12]{figs/networks/corn2017usneighbors.png}
            \end{figure}
        \end{minipage}
    \end{column}
\hfill
    \begin{column}{.5\textwidth}
        \begin{minipage}[t][.5\textheight][t]{\textwidth}
            \begin{figure}
                \caption*{\tiny US neighbors (2019)}
                \includegraphics[scale=0.11]{figs/networks/corn2019usneighbors.png}\\
            \end{figure}
        \end{minipage}
    \end{column}
\end{columns}
\vspace{-35pt}
\begin{figure}
    \centering
    \caption*{\tiny US neighbors (2022)}
    \includegraphics[scale=0.15]{figs/networks/corn2022usneighbors.png}    
\end{figure}

\end{frame}
\begin{frame}{Communities - Corn (Brazil)}
\begin{columns}[onlytextwidth]
    \begin{column}{.5\textwidth}
        \begin{minipage}[t][.5\textheight][t]{\textwidth}
            \begin{figure}
                \caption*{\tiny Brazil neighbors (2017)}
                \includegraphics[scale=0.12]{figs/networks/corn2017brazilneighbors.png}
            \end{figure}
        \end{minipage}
    \end{column}
\hfill
    \begin{column}{.5\textwidth}
        \begin{minipage}[t][.5\textheight][t]{\textwidth}
            \begin{figure}
                \caption*{\tiny Brazil neighbors (2019)}
                \includegraphics[scale=0.13]{figs/networks/corn2022brazilneighbor.png}\\
            \end{figure}
        \end{minipage}
    \end{column}
\end{columns}
\vspace{-35pt}
\begin{figure}
    \centering
    \caption*{\tiny Brazil neighbors (2022)}
    \includegraphics[scale=0.16]{figs/networks/pork2022brazilneighbors.png}    
\end{figure}

\end{frame}
\fi

\begin{frame}{Clustering (Corn)}
\begin{figure}
    \centering
    \includegraphics[scale=0.7]{figs/presentationFigs/corn-clust.pdf}
\end{figure}
\begin{itemize}
    \small
    \item[\ding{213}] Declining clustering coefficient, similar to pork and other commodities.  \\
    
\end{itemize}
\end{frame}
\begin{frame}{Clustering}
    \includegraphics[width=\textwidth]{figs/presentationFigs/clust-All.pdf}
\end{frame}
\section{Takeaways}
\begin{frame}{Takeaways and Future}
    \begin{itemize}\itemsep20pt
        \item[\ding{213}] Food supply chains are changing: new influencers, more sub-chains, tendency towards deglobalization. 
        
        \item[\ding{213}] Friendshoring in early stages...
        \item[\ding{213}] Resiliency and safeguards are major thrust areas.
        \item[\ding{213}] Food security is under threat if supply chains deglobalize (availability and access issues).
    \end{itemize}
\end{frame}
\appendix

\begin{frame}[plain]
    \Large Appendix

\end{frame}
\begin{frame}{Centrality Measures}
    \includegraphics[scale=0.2]{figs/presentationFigs/centralitiesAll.pdf}
\end{frame}
\begin{frame}{Community Detection}
    \includegraphics[width=\textwidth]{figs/presentationFigs/CommunitiesAllComms.png}
\end{frame}

\begin{frame}{Clustering}
    \includegraphics[width=\textwidth]{figs/presentationFigs/clust-All.pdf}
\end{frame}
